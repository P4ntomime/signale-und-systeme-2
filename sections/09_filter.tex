\section{Filter}


\subsection{Grundtypen}{291}

Filter sind mehrheitlich \textbf{frequnezselektive, lineare Netzwerke}, welche gewisse Frequenzbereiche übertragen
und andere dämpfen. Die fünf \textbf{frequnezselektiven Grundtypen} sind: 

\begin{minipage}[t]{0.25\columnwidth}
    \begin{itemize}
        \item Tiefpass (TP)
        \item Hochpass (HP)
    \end{itemize}
\end{minipage}
\hfill
\begin{minipage}[t]{0.35\columnwidth}
    \begin{itemize}
        \item Bandpass (BP)
        \item Bandsperre, Notch (BS)
    \end{itemize}
\end{minipage}
\hfill
\begin{minipage}[t]{0.25\columnwidth}
    \begin{itemize}
        \item Allpass
    \end{itemize}
\end{minipage}


\subsection{Filterbauarten}{292}

\textrightarrow\ Siehe Skript S. 292

% \begin{tabular}{|l|l|l|l|}
%     \hline
%     \multirow{7}{*}{\rotatebox[origin=c]{90}{zeitdiskret}}         & \multirow{7}{*}{aktiv}    & \multirow{3}{*}{\begin{tabular}[c]{@{}l@{}}diskrete\\ Werte\end{tabular}}         & Filter mit Signalprozessor           \\ \cline{4-4} 
%                                       &                            &                                                                                   & Filter mit Gate Logic                \\ \cline{4-4} 
%                                       &                            &                                                                                   & etc.                                 \\ \cline{3-4} 
%                                       &                            & \multirow{4}{*}{\begin{tabular}[c]{@{}l@{}}analoge\\ Werte\end{tabular}}          & SC-Filter (Switched Capacitor)       \\ \cline{4-4} 
%                                       &                            &                                                                                   & SI-Filter (Switched Current)         \\ \cline{4-4} 
%                                       &                            &                                                                                   & N-Pfad Filter                        \\ \cline{4-4} 
%                                       &                            &                                                                                   & etc.                                 \\ \hline
%     \multirow{16}{*}{\rotatebox[origin=c]{90}{kontinurierlich}}    & \multirow{5}{*}{aktive RC} & \multirow{5}{*}{\begin{tabular}[c]{@{}l@{}}konz. \\ Bauteile\end{tabular}}        & 'Simultion' von LC-Filtern           \\ \cline{4-4} 
%                                       &                            &                                                                                   & Gekoppelte Filterstrukturen          \\ \cline{4-4} 
%                                       &                            &                                                                                   & Filter Parallelform                  \\ \cline{4-4} 
%                                       &                            &                                                                                   & Filter in Kaskadenbauweise           \\ \cline{4-4} 
%                                       &                            &                                                                                   & etc.                                 \\ \cline{2-4} 
%                                       & \multirow{11}{*}{passiv}   & \multirow{4}{*}{\begin{tabular}[c]{@{}l@{}}verteilte\\ Bauteile\end{tabular}}     & SAW-Filter (Surfaec Acoustic Wave)   \\ \cline{4-4} 
%                                       &                            &                                                                                   & Hohlraumresonatoren (Topfkreisfilter \\ \cline{4-4} 
%                                       &                            &                                                                                   & Leisutngsfilter                      \\ \cline{4-4} 
%                                       &                            &                                                                                   & etc.                                 \\ \cline{3-4} 
%                                       &                            & \multirow{7}{*}{\begin{tabular}[c]{@{}l@{}}konzentrierte\\ Bauteile\end{tabular}} & Keramische Filter                    \\ \cline{4-4} 
%                                       &                            &                                                                                   & Quarz-Filter                         \\ \cline{4-4} 
%                                       &                            &                                                                                   & RLC-Filter                           \\ \cline{4-4} 
%                                       &                            &                                                                                   & RC-Filter                            \\ \cline{4-4} 
%                                       &                            &                                                                                   & LC-Filter                            \\ \cline{4-4} 
%                                       &                            &                                                                                   & LC-Filter mit Quarz                  \\ \cline{4-4} 
%                                       &                            &                                                                                   & etc.                                 \\ \hline
% \end{tabular}


\subsection{Ideales Tiefpassfilter}{297}

unendlich lange Impulsantwort, akausal -> nicht realisierbar





\subsection{Filterspezifikation (Stempel und Matritze)}{297}

Amplitudengang soll zwischen den Rechtecken durch gehen
-> normiert auf $\Omega$

Normierungsgrössen als Formel

Entnormierung $S$ in normierter Funktion durch $\frac{s}{\omega_r}$ ersetzen

\subsection{RC-Filer 1. Ordnung}    % In Video von Mathis Woche 11