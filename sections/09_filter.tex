\section{Filter}


\subsection{Grundtypen}{291}

Filter sind mehrheitlich \textbf{frequnezselektive, lineare Netzwerke}, welche gewisse Frequenzbereiche übertragen
und andere dämpfen. Die fünf \textbf{frequnezselektiven Grundtypen} sind: 

\begin{minipage}[t]{0.25\columnwidth}
    \begin{itemize}
        \item Tiefpass (TP)
        \item Hochpass (HP)
    \end{itemize}
\end{minipage}
\hfill
\begin{minipage}[t]{0.35\columnwidth}
    \begin{itemize}
        \item Bandpass (BP)
        \item Bandsperre, Notch (BS)
    \end{itemize}
\end{minipage}
\hfill
\begin{minipage}[t]{0.25\columnwidth}
    \begin{itemize}
        \item Allpass
    \end{itemize}
\end{minipage}


% \subsection{Filterbauarten}{292}

% \textrightarrow\ Siehe Skript S. 292

% \begin{tabular}{|l|l|l|l|}
%     \hline
%     \multirow{7}{*}{\rotatebox[origin=c]{90}{zeitdiskret}}         & \multirow{7}{*}{aktiv}    & \multirow{3}{*}{\begin{tabular}[c]{@{}l@{}}diskrete\\ Werte\end{tabular}}         & Filter mit Signalprozessor           \\ \cline{4-4} 
%                                       &                            &                                                                                   & Filter mit Gate Logic                \\ \cline{4-4} 
%                                       &                            &                                                                                   & etc.                                 \\ \cline{3-4} 
%                                       &                            & \multirow{4}{*}{\begin{tabular}[c]{@{}l@{}}analoge\\ Werte\end{tabular}}          & SC-Filter (Switched Capacitor)       \\ \cline{4-4} 
%                                       &                            &                                                                                   & SI-Filter (Switched Current)         \\ \cline{4-4} 
%                                       &                            &                                                                                   & N-Pfad Filter                        \\ \cline{4-4} 
%                                       &                            &                                                                                   & etc.                                 \\ \hline
%     \multirow{16}{*}{\rotatebox[origin=c]{90}{kontinurierlich}}    & \multirow{5}{*}{aktive RC} & \multirow{5}{*}{\begin{tabular}[c]{@{}l@{}}konz. \\ Bauteile\end{tabular}}        & 'Simultion' von LC-Filtern           \\ \cline{4-4} 
%                                       &                            &                                                                                   & Gekoppelte Filterstrukturen          \\ \cline{4-4} 
%                                       &                            &                                                                                   & Filter Parallelform                  \\ \cline{4-4} 
%                                       &                            &                                                                                   & Filter in Kaskadenbauweise           \\ \cline{4-4} 
%                                       &                            &                                                                                   & etc.                                 \\ \cline{2-4} 
%                                       & \multirow{11}{*}{passiv}   & \multirow{4}{*}{\begin{tabular}[c]{@{}l@{}}verteilte\\ Bauteile\end{tabular}}     & SAW-Filter (Surfaec Acoustic Wave)   \\ \cline{4-4} 
%                                       &                            &                                                                                   & Hohlraumresonatoren (Topfkreisfilter \\ \cline{4-4} 
%                                       &                            &                                                                                   & Leisutngsfilter                      \\ \cline{4-4} 
%                                       &                            &                                                                                   & etc.                                 \\ \cline{3-4} 
%                                       &                            & \multirow{7}{*}{\begin{tabular}[c]{@{}l@{}}konzentrierte\\ Bauteile\end{tabular}} & Keramische Filter                    \\ \cline{4-4} 
%                                       &                            &                                                                                   & Quarz-Filter                         \\ \cline{4-4} 
%                                       &                            &                                                                                   & RLC-Filter                           \\ \cline{4-4} 
%                                       &                            &                                                                                   & RC-Filter                            \\ \cline{4-4} 
%                                       &                            &                                                                                   & LC-Filter                            \\ \cline{4-4} 
%                                       &                            &                                                                                   & LC-Filter mit Quarz                  \\ \cline{4-4} 
%                                       &                            &                                                                                   & etc.                                 \\ \hline
% \end{tabular}


\subsection[Frequnezgang H(jimg omega) -- Übertragungsfunktion H(s)]{Frequnezgang $H(\jimg \omega)$ -- Übertragungsfunktion $H(s)$}{294}

Für den Frequnezgang $H(\jimg \omega)$ und die Übertragungsfunktion $H(s)$ gelten die folgenden Zusammenhänge

$$ | H(\jimg \omega) |^2 = H(\jimg \omega) \cdot H^*(\jimg \omega) = H(\jimg \omega) \cdot H(- \jimg \omega) = H(s) \cdot H(-s) \Big|_{s = \jimg \omega} $$
$$ H(s) \cdot H(-s) = | H(\jimg \omega) |^2 \Big|_{\omega^2 = -s^2} $$
\textbf{Hinweis:} $| H(\jimg \omega) |^2$ ist immer eine Funktion in $\omega^2$, da der Amplitudengang eine gerade Funktion ist!

\vspace{0.2cm}

Da in der Praxis \textbf{jeweils nur $\bm{H(s)}$ interessant} ist, muss $H(s)$ aus $| H(\jimg \omega) |^2$ 'isoliert' werden. 
Dies ist durch den folgenden Zusammenhang möglich.

$$ \boxed{ \underbrace{ \frac{N(s)}{D(s)} }_{H(s)} \cdot  \underbrace{ \frac{N(-s)}{D(-s)} }_{H(-s)} = | H(\jimg \omega) |^2 \Big|_{\omega^2 = -s^2} } $$
\textbf{Hinweis:} $D(s)$ muss aus Stabilitätsgründen ein Hurwitz-Polynom sein!


\subsection{Approximation im Frequnezbereich}

Die wichtigste Aufgabe der Filtertheorie ist die \textbf{Bestimmung der Übertragungsfunktion, die einen vorgegebenen 
Frequenzgang gewährleistet.} Zuerst soll der \textbf{Amplitudengang} $| H(\jimg \omega) |$ im Frequnezbereich approximiert werden.
Der vorgeschriebene Phasengang wird dann allenfalls mit zusätzlichen Allpass-Filtern erreicht. 


\subsubsection{Toleranzschema (Stempel und Matritze) -- Filterspezifikation}



\subsection{Ideales Tiefpassfilter}{297}

unendlich lange Impulsantwort, akausal -> nicht realisierbar




\subsection{Filterspezifikation (Stempel und Matritze)}{297}

Amplitudengang soll zwischen den Rechtecken durch gehen
-> normiert auf $\Omega$

Normierungsgrössen als Formel

Entnormierung $S$ in normierter Funktion durch $\frac{s}{\omega_r}$ ersetzen

\subsection{RC-Filter 1. Ordnung}    % In Video von Mathis Woche 11


\subsection{Standard-Filtertypen}

% gemäss Praktikum Vorteile und Nachteile von jedem Filter auflisten


\subsection{Approximation nach Butterworth}